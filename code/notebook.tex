
% Default to the notebook output style

    


% Inherit from the specified cell style.




    
\documentclass[11pt]{article}

    
    
    \usepackage[T1]{fontenc}
    % Nicer default font (+ math font) than Computer Modern for most use cases
    \usepackage{mathpazo}

    % Basic figure setup, for now with no caption control since it's done
    % automatically by Pandoc (which extracts ![](path) syntax from Markdown).
    \usepackage{graphicx}
    % We will generate all images so they have a width \maxwidth. This means
    % that they will get their normal width if they fit onto the page, but
    % are scaled down if they would overflow the margins.
    \makeatletter
    \def\maxwidth{\ifdim\Gin@nat@width>\linewidth\linewidth
    \else\Gin@nat@width\fi}
    \makeatother
    \let\Oldincludegraphics\includegraphics
    % Set max figure width to be 80% of text width, for now hardcoded.
    \renewcommand{\includegraphics}[1]{\Oldincludegraphics[width=.8\maxwidth]{#1}}
    % Ensure that by default, figures have no caption (until we provide a
    % proper Figure object with a Caption API and a way to capture that
    % in the conversion process - todo).
    \usepackage{caption}
    \DeclareCaptionLabelFormat{nolabel}{}
    \captionsetup{labelformat=nolabel}

    \usepackage{adjustbox} % Used to constrain images to a maximum size 
    \usepackage{xcolor} % Allow colors to be defined
    \usepackage{enumerate} % Needed for markdown enumerations to work
    \usepackage{geometry} % Used to adjust the document margins
    \usepackage{amsmath} % Equations
    \usepackage{amssymb} % Equations
    \usepackage{textcomp} % defines textquotesingle
    % Hack from http://tex.stackexchange.com/a/47451/13684:
    \AtBeginDocument{%
        \def\PYZsq{\textquotesingle}% Upright quotes in Pygmentized code
    }
    \usepackage{upquote} % Upright quotes for verbatim code
    \usepackage{eurosym} % defines \euro
    \usepackage[mathletters]{ucs} % Extended unicode (utf-8) support
    \usepackage[utf8x]{inputenc} % Allow utf-8 characters in the tex document
    \usepackage{fancyvrb} % verbatim replacement that allows latex
    \usepackage{grffile} % extends the file name processing of package graphics 
                         % to support a larger range 
    % The hyperref package gives us a pdf with properly built
    % internal navigation ('pdf bookmarks' for the table of contents,
    % internal cross-reference links, web links for URLs, etc.)
    \usepackage{hyperref}
    \usepackage{longtable} % longtable support required by pandoc >1.10
    \usepackage{booktabs}  % table support for pandoc > 1.12.2
    \usepackage[inline]{enumitem} % IRkernel/repr support (it uses the enumerate* environment)
    \usepackage[normalem]{ulem} % ulem is needed to support strikethroughs (\sout)
                                % normalem makes italics be italics, not underlines
    

    
    
    % Colors for the hyperref package
    \definecolor{urlcolor}{rgb}{0,.145,.698}
    \definecolor{linkcolor}{rgb}{.71,0.21,0.01}
    \definecolor{citecolor}{rgb}{.12,.54,.11}

    % ANSI colors
    \definecolor{ansi-black}{HTML}{3E424D}
    \definecolor{ansi-black-intense}{HTML}{282C36}
    \definecolor{ansi-red}{HTML}{E75C58}
    \definecolor{ansi-red-intense}{HTML}{B22B31}
    \definecolor{ansi-green}{HTML}{00A250}
    \definecolor{ansi-green-intense}{HTML}{007427}
    \definecolor{ansi-yellow}{HTML}{DDB62B}
    \definecolor{ansi-yellow-intense}{HTML}{B27D12}
    \definecolor{ansi-blue}{HTML}{208FFB}
    \definecolor{ansi-blue-intense}{HTML}{0065CA}
    \definecolor{ansi-magenta}{HTML}{D160C4}
    \definecolor{ansi-magenta-intense}{HTML}{A03196}
    \definecolor{ansi-cyan}{HTML}{60C6C8}
    \definecolor{ansi-cyan-intense}{HTML}{258F8F}
    \definecolor{ansi-white}{HTML}{C5C1B4}
    \definecolor{ansi-white-intense}{HTML}{A1A6B2}

    % commands and environments needed by pandoc snippets
    % extracted from the output of `pandoc -s`
    \providecommand{\tightlist}{%
      \setlength{\itemsep}{0pt}\setlength{\parskip}{0pt}}
    \DefineVerbatimEnvironment{Highlighting}{Verbatim}{commandchars=\\\{\}}
    % Add ',fontsize=\small' for more characters per line
    \newenvironment{Shaded}{}{}
    \newcommand{\KeywordTok}[1]{\textcolor[rgb]{0.00,0.44,0.13}{\textbf{{#1}}}}
    \newcommand{\DataTypeTok}[1]{\textcolor[rgb]{0.56,0.13,0.00}{{#1}}}
    \newcommand{\DecValTok}[1]{\textcolor[rgb]{0.25,0.63,0.44}{{#1}}}
    \newcommand{\BaseNTok}[1]{\textcolor[rgb]{0.25,0.63,0.44}{{#1}}}
    \newcommand{\FloatTok}[1]{\textcolor[rgb]{0.25,0.63,0.44}{{#1}}}
    \newcommand{\CharTok}[1]{\textcolor[rgb]{0.25,0.44,0.63}{{#1}}}
    \newcommand{\StringTok}[1]{\textcolor[rgb]{0.25,0.44,0.63}{{#1}}}
    \newcommand{\CommentTok}[1]{\textcolor[rgb]{0.38,0.63,0.69}{\textit{{#1}}}}
    \newcommand{\OtherTok}[1]{\textcolor[rgb]{0.00,0.44,0.13}{{#1}}}
    \newcommand{\AlertTok}[1]{\textcolor[rgb]{1.00,0.00,0.00}{\textbf{{#1}}}}
    \newcommand{\FunctionTok}[1]{\textcolor[rgb]{0.02,0.16,0.49}{{#1}}}
    \newcommand{\RegionMarkerTok}[1]{{#1}}
    \newcommand{\ErrorTok}[1]{\textcolor[rgb]{1.00,0.00,0.00}{\textbf{{#1}}}}
    \newcommand{\NormalTok}[1]{{#1}}
    
    % Additional commands for more recent versions of Pandoc
    \newcommand{\ConstantTok}[1]{\textcolor[rgb]{0.53,0.00,0.00}{{#1}}}
    \newcommand{\SpecialCharTok}[1]{\textcolor[rgb]{0.25,0.44,0.63}{{#1}}}
    \newcommand{\VerbatimStringTok}[1]{\textcolor[rgb]{0.25,0.44,0.63}{{#1}}}
    \newcommand{\SpecialStringTok}[1]{\textcolor[rgb]{0.73,0.40,0.53}{{#1}}}
    \newcommand{\ImportTok}[1]{{#1}}
    \newcommand{\DocumentationTok}[1]{\textcolor[rgb]{0.73,0.13,0.13}{\textit{{#1}}}}
    \newcommand{\AnnotationTok}[1]{\textcolor[rgb]{0.38,0.63,0.69}{\textbf{\textit{{#1}}}}}
    \newcommand{\CommentVarTok}[1]{\textcolor[rgb]{0.38,0.63,0.69}{\textbf{\textit{{#1}}}}}
    \newcommand{\VariableTok}[1]{\textcolor[rgb]{0.10,0.09,0.49}{{#1}}}
    \newcommand{\ControlFlowTok}[1]{\textcolor[rgb]{0.00,0.44,0.13}{\textbf{{#1}}}}
    \newcommand{\OperatorTok}[1]{\textcolor[rgb]{0.40,0.40,0.40}{{#1}}}
    \newcommand{\BuiltInTok}[1]{{#1}}
    \newcommand{\ExtensionTok}[1]{{#1}}
    \newcommand{\PreprocessorTok}[1]{\textcolor[rgb]{0.74,0.48,0.00}{{#1}}}
    \newcommand{\AttributeTok}[1]{\textcolor[rgb]{0.49,0.56,0.16}{{#1}}}
    \newcommand{\InformationTok}[1]{\textcolor[rgb]{0.38,0.63,0.69}{\textbf{\textit{{#1}}}}}
    \newcommand{\WarningTok}[1]{\textcolor[rgb]{0.38,0.63,0.69}{\textbf{\textit{{#1}}}}}
    
    
    % Define a nice break command that doesn't care if a line doesn't already
    % exist.
    \def\br{\hspace*{\fill} \\* }
    % Math Jax compatability definitions
    \def\gt{>}
    \def\lt{<}
    % Document parameters
    \title{PES\_VV\_v1}
    
    
    

    % Pygments definitions
    
\makeatletter
\def\PY@reset{\let\PY@it=\relax \let\PY@bf=\relax%
    \let\PY@ul=\relax \let\PY@tc=\relax%
    \let\PY@bc=\relax \let\PY@ff=\relax}
\def\PY@tok#1{\csname PY@tok@#1\endcsname}
\def\PY@toks#1+{\ifx\relax#1\empty\else%
    \PY@tok{#1}\expandafter\PY@toks\fi}
\def\PY@do#1{\PY@bc{\PY@tc{\PY@ul{%
    \PY@it{\PY@bf{\PY@ff{#1}}}}}}}
\def\PY#1#2{\PY@reset\PY@toks#1+\relax+\PY@do{#2}}

\expandafter\def\csname PY@tok@w\endcsname{\def\PY@tc##1{\textcolor[rgb]{0.73,0.73,0.73}{##1}}}
\expandafter\def\csname PY@tok@c\endcsname{\let\PY@it=\textit\def\PY@tc##1{\textcolor[rgb]{0.25,0.50,0.50}{##1}}}
\expandafter\def\csname PY@tok@cp\endcsname{\def\PY@tc##1{\textcolor[rgb]{0.74,0.48,0.00}{##1}}}
\expandafter\def\csname PY@tok@k\endcsname{\let\PY@bf=\textbf\def\PY@tc##1{\textcolor[rgb]{0.00,0.50,0.00}{##1}}}
\expandafter\def\csname PY@tok@kp\endcsname{\def\PY@tc##1{\textcolor[rgb]{0.00,0.50,0.00}{##1}}}
\expandafter\def\csname PY@tok@kt\endcsname{\def\PY@tc##1{\textcolor[rgb]{0.69,0.00,0.25}{##1}}}
\expandafter\def\csname PY@tok@o\endcsname{\def\PY@tc##1{\textcolor[rgb]{0.40,0.40,0.40}{##1}}}
\expandafter\def\csname PY@tok@ow\endcsname{\let\PY@bf=\textbf\def\PY@tc##1{\textcolor[rgb]{0.67,0.13,1.00}{##1}}}
\expandafter\def\csname PY@tok@nb\endcsname{\def\PY@tc##1{\textcolor[rgb]{0.00,0.50,0.00}{##1}}}
\expandafter\def\csname PY@tok@nf\endcsname{\def\PY@tc##1{\textcolor[rgb]{0.00,0.00,1.00}{##1}}}
\expandafter\def\csname PY@tok@nc\endcsname{\let\PY@bf=\textbf\def\PY@tc##1{\textcolor[rgb]{0.00,0.00,1.00}{##1}}}
\expandafter\def\csname PY@tok@nn\endcsname{\let\PY@bf=\textbf\def\PY@tc##1{\textcolor[rgb]{0.00,0.00,1.00}{##1}}}
\expandafter\def\csname PY@tok@ne\endcsname{\let\PY@bf=\textbf\def\PY@tc##1{\textcolor[rgb]{0.82,0.25,0.23}{##1}}}
\expandafter\def\csname PY@tok@nv\endcsname{\def\PY@tc##1{\textcolor[rgb]{0.10,0.09,0.49}{##1}}}
\expandafter\def\csname PY@tok@no\endcsname{\def\PY@tc##1{\textcolor[rgb]{0.53,0.00,0.00}{##1}}}
\expandafter\def\csname PY@tok@nl\endcsname{\def\PY@tc##1{\textcolor[rgb]{0.63,0.63,0.00}{##1}}}
\expandafter\def\csname PY@tok@ni\endcsname{\let\PY@bf=\textbf\def\PY@tc##1{\textcolor[rgb]{0.60,0.60,0.60}{##1}}}
\expandafter\def\csname PY@tok@na\endcsname{\def\PY@tc##1{\textcolor[rgb]{0.49,0.56,0.16}{##1}}}
\expandafter\def\csname PY@tok@nt\endcsname{\let\PY@bf=\textbf\def\PY@tc##1{\textcolor[rgb]{0.00,0.50,0.00}{##1}}}
\expandafter\def\csname PY@tok@nd\endcsname{\def\PY@tc##1{\textcolor[rgb]{0.67,0.13,1.00}{##1}}}
\expandafter\def\csname PY@tok@s\endcsname{\def\PY@tc##1{\textcolor[rgb]{0.73,0.13,0.13}{##1}}}
\expandafter\def\csname PY@tok@sd\endcsname{\let\PY@it=\textit\def\PY@tc##1{\textcolor[rgb]{0.73,0.13,0.13}{##1}}}
\expandafter\def\csname PY@tok@si\endcsname{\let\PY@bf=\textbf\def\PY@tc##1{\textcolor[rgb]{0.73,0.40,0.53}{##1}}}
\expandafter\def\csname PY@tok@se\endcsname{\let\PY@bf=\textbf\def\PY@tc##1{\textcolor[rgb]{0.73,0.40,0.13}{##1}}}
\expandafter\def\csname PY@tok@sr\endcsname{\def\PY@tc##1{\textcolor[rgb]{0.73,0.40,0.53}{##1}}}
\expandafter\def\csname PY@tok@ss\endcsname{\def\PY@tc##1{\textcolor[rgb]{0.10,0.09,0.49}{##1}}}
\expandafter\def\csname PY@tok@sx\endcsname{\def\PY@tc##1{\textcolor[rgb]{0.00,0.50,0.00}{##1}}}
\expandafter\def\csname PY@tok@m\endcsname{\def\PY@tc##1{\textcolor[rgb]{0.40,0.40,0.40}{##1}}}
\expandafter\def\csname PY@tok@gh\endcsname{\let\PY@bf=\textbf\def\PY@tc##1{\textcolor[rgb]{0.00,0.00,0.50}{##1}}}
\expandafter\def\csname PY@tok@gu\endcsname{\let\PY@bf=\textbf\def\PY@tc##1{\textcolor[rgb]{0.50,0.00,0.50}{##1}}}
\expandafter\def\csname PY@tok@gd\endcsname{\def\PY@tc##1{\textcolor[rgb]{0.63,0.00,0.00}{##1}}}
\expandafter\def\csname PY@tok@gi\endcsname{\def\PY@tc##1{\textcolor[rgb]{0.00,0.63,0.00}{##1}}}
\expandafter\def\csname PY@tok@gr\endcsname{\def\PY@tc##1{\textcolor[rgb]{1.00,0.00,0.00}{##1}}}
\expandafter\def\csname PY@tok@ge\endcsname{\let\PY@it=\textit}
\expandafter\def\csname PY@tok@gs\endcsname{\let\PY@bf=\textbf}
\expandafter\def\csname PY@tok@gp\endcsname{\let\PY@bf=\textbf\def\PY@tc##1{\textcolor[rgb]{0.00,0.00,0.50}{##1}}}
\expandafter\def\csname PY@tok@go\endcsname{\def\PY@tc##1{\textcolor[rgb]{0.53,0.53,0.53}{##1}}}
\expandafter\def\csname PY@tok@gt\endcsname{\def\PY@tc##1{\textcolor[rgb]{0.00,0.27,0.87}{##1}}}
\expandafter\def\csname PY@tok@err\endcsname{\def\PY@bc##1{\setlength{\fboxsep}{0pt}\fcolorbox[rgb]{1.00,0.00,0.00}{1,1,1}{\strut ##1}}}
\expandafter\def\csname PY@tok@kc\endcsname{\let\PY@bf=\textbf\def\PY@tc##1{\textcolor[rgb]{0.00,0.50,0.00}{##1}}}
\expandafter\def\csname PY@tok@kd\endcsname{\let\PY@bf=\textbf\def\PY@tc##1{\textcolor[rgb]{0.00,0.50,0.00}{##1}}}
\expandafter\def\csname PY@tok@kn\endcsname{\let\PY@bf=\textbf\def\PY@tc##1{\textcolor[rgb]{0.00,0.50,0.00}{##1}}}
\expandafter\def\csname PY@tok@kr\endcsname{\let\PY@bf=\textbf\def\PY@tc##1{\textcolor[rgb]{0.00,0.50,0.00}{##1}}}
\expandafter\def\csname PY@tok@bp\endcsname{\def\PY@tc##1{\textcolor[rgb]{0.00,0.50,0.00}{##1}}}
\expandafter\def\csname PY@tok@fm\endcsname{\def\PY@tc##1{\textcolor[rgb]{0.00,0.00,1.00}{##1}}}
\expandafter\def\csname PY@tok@vc\endcsname{\def\PY@tc##1{\textcolor[rgb]{0.10,0.09,0.49}{##1}}}
\expandafter\def\csname PY@tok@vg\endcsname{\def\PY@tc##1{\textcolor[rgb]{0.10,0.09,0.49}{##1}}}
\expandafter\def\csname PY@tok@vi\endcsname{\def\PY@tc##1{\textcolor[rgb]{0.10,0.09,0.49}{##1}}}
\expandafter\def\csname PY@tok@vm\endcsname{\def\PY@tc##1{\textcolor[rgb]{0.10,0.09,0.49}{##1}}}
\expandafter\def\csname PY@tok@sa\endcsname{\def\PY@tc##1{\textcolor[rgb]{0.73,0.13,0.13}{##1}}}
\expandafter\def\csname PY@tok@sb\endcsname{\def\PY@tc##1{\textcolor[rgb]{0.73,0.13,0.13}{##1}}}
\expandafter\def\csname PY@tok@sc\endcsname{\def\PY@tc##1{\textcolor[rgb]{0.73,0.13,0.13}{##1}}}
\expandafter\def\csname PY@tok@dl\endcsname{\def\PY@tc##1{\textcolor[rgb]{0.73,0.13,0.13}{##1}}}
\expandafter\def\csname PY@tok@s2\endcsname{\def\PY@tc##1{\textcolor[rgb]{0.73,0.13,0.13}{##1}}}
\expandafter\def\csname PY@tok@sh\endcsname{\def\PY@tc##1{\textcolor[rgb]{0.73,0.13,0.13}{##1}}}
\expandafter\def\csname PY@tok@s1\endcsname{\def\PY@tc##1{\textcolor[rgb]{0.73,0.13,0.13}{##1}}}
\expandafter\def\csname PY@tok@mb\endcsname{\def\PY@tc##1{\textcolor[rgb]{0.40,0.40,0.40}{##1}}}
\expandafter\def\csname PY@tok@mf\endcsname{\def\PY@tc##1{\textcolor[rgb]{0.40,0.40,0.40}{##1}}}
\expandafter\def\csname PY@tok@mh\endcsname{\def\PY@tc##1{\textcolor[rgb]{0.40,0.40,0.40}{##1}}}
\expandafter\def\csname PY@tok@mi\endcsname{\def\PY@tc##1{\textcolor[rgb]{0.40,0.40,0.40}{##1}}}
\expandafter\def\csname PY@tok@il\endcsname{\def\PY@tc##1{\textcolor[rgb]{0.40,0.40,0.40}{##1}}}
\expandafter\def\csname PY@tok@mo\endcsname{\def\PY@tc##1{\textcolor[rgb]{0.40,0.40,0.40}{##1}}}
\expandafter\def\csname PY@tok@ch\endcsname{\let\PY@it=\textit\def\PY@tc##1{\textcolor[rgb]{0.25,0.50,0.50}{##1}}}
\expandafter\def\csname PY@tok@cm\endcsname{\let\PY@it=\textit\def\PY@tc##1{\textcolor[rgb]{0.25,0.50,0.50}{##1}}}
\expandafter\def\csname PY@tok@cpf\endcsname{\let\PY@it=\textit\def\PY@tc##1{\textcolor[rgb]{0.25,0.50,0.50}{##1}}}
\expandafter\def\csname PY@tok@c1\endcsname{\let\PY@it=\textit\def\PY@tc##1{\textcolor[rgb]{0.25,0.50,0.50}{##1}}}
\expandafter\def\csname PY@tok@cs\endcsname{\let\PY@it=\textit\def\PY@tc##1{\textcolor[rgb]{0.25,0.50,0.50}{##1}}}

\def\PYZbs{\char`\\}
\def\PYZus{\char`\_}
\def\PYZob{\char`\{}
\def\PYZcb{\char`\}}
\def\PYZca{\char`\^}
\def\PYZam{\char`\&}
\def\PYZlt{\char`\<}
\def\PYZgt{\char`\>}
\def\PYZsh{\char`\#}
\def\PYZpc{\char`\%}
\def\PYZdl{\char`\$}
\def\PYZhy{\char`\-}
\def\PYZsq{\char`\'}
\def\PYZdq{\char`\"}
\def\PYZti{\char`\~}
% for compatibility with earlier versions
\def\PYZat{@}
\def\PYZlb{[}
\def\PYZrb{]}
\makeatother


    % Exact colors from NB
    \definecolor{incolor}{rgb}{0.0, 0.0, 0.5}
    \definecolor{outcolor}{rgb}{0.545, 0.0, 0.0}



    
    % Prevent overflowing lines due to hard-to-break entities
    \sloppy 
    % Setup hyperref package
    \hypersetup{
      breaklinks=true,  % so long urls are correctly broken across lines
      colorlinks=true,
      urlcolor=urlcolor,
      linkcolor=linkcolor,
      citecolor=citecolor,
      }
    % Slightly bigger margins than the latex defaults
    
    \geometry{verbose,tmargin=1in,bmargin=1in,lmargin=1in,rmargin=1in}
    
    

    \begin{document}
    
    
    \maketitle
    
    

    
    \section{\texorpdfstring{\emph{Ab} \emph{initio} molecular dynamics of
the vibrational motion of
HF}{Ab initio molecular dynamics of the vibrational motion of HF}}\label{ab-initio-molecular-dynamics-of-the-vibrational-motion-of-hf}

We are going to construct what is often referred to as an \emph{ab}
\emph{initio} potential energy surface of the diatomic molecule hydrogen
fluoride. That is, we are going to use various electronic structure
theories (Hartree-Fock theory (RHF), 2nd-order perturbation theory
(MP2), and Coupled Cluster theory with single and double substitutions
(CCSD)) to compute the electronic energy at different geometries of a
simple diatomic molecule. The same basis set (correlation consistent
polarized triple-zeta, cc-pVTZ) will be used for all calculations. We
will use Psi4numpy to facilitate the electronic structure calculations,
and then the interpolation capabilities of scipy to simplify the
evalution of the potential energy at separations for which we did not
explicitly evaluate the electronic energy. We will also use scipy to
differentiate the interpolated potential energy surface to obtain the
forces acting on the atoms at different separations.

We will start by importing the necessary libraries:

    \begin{Verbatim}[commandchars=\\\{\}]
{\color{incolor}In [{\color{incolor}1}]:} \PY{k+kn}{import} \PY{n+nn}{numpy} \PY{k}{as} \PY{n+nn}{np}
        \PY{k+kn}{import} \PY{n+nn}{psi4}
        \PY{k+kn}{from} \PY{n+nn}{matplotlib} \PY{k}{import} \PY{n}{pyplot} \PY{k}{as} \PY{n}{plt}
        \PY{k+kn}{from} \PY{n+nn}{scipy}\PY{n+nn}{.}\PY{n+nn}{interpolate} \PY{k}{import} \PY{n}{InterpolatedUnivariateSpline}
\end{Verbatim}


    \begin{Verbatim}[commandchars=\\\{\}]
{\color{incolor}In [{\color{incolor}16}]:} \PY{c+c1}{\PYZsh{}\PYZsh{}\PYZsh{} template for the z\PYZhy{}matrix}
         \PY{n}{mol\PYZus{}tmpl} \PY{o}{=} \PY{l+s+s2}{\PYZdq{}\PYZdq{}\PYZdq{}}\PY{l+s+s2}{H}
         \PY{l+s+s2}{F 1 **R**}\PY{l+s+s2}{\PYZdq{}\PYZdq{}\PYZdq{}}
         \PY{c+c1}{\PYZsh{}\PYZsh{}\PYZsh{} array of bond\PYZhy{}lengths in anstromgs}
         \PY{n}{r\PYZus{}array} \PY{o}{=} \PY{n}{np}\PY{o}{.}\PY{n}{array}\PY{p}{(}\PY{p}{[}\PY{l+m+mf}{0.5}\PY{p}{,} \PY{l+m+mf}{0.55}\PY{p}{,} \PY{l+m+mf}{0.6}\PY{p}{,} \PY{l+m+mf}{0.65}\PY{p}{,} \PY{l+m+mf}{0.7}\PY{p}{,} \PY{l+m+mf}{0.75}\PY{p}{,} \PY{l+m+mf}{0.8}\PY{p}{,} \PY{l+m+mf}{0.85}\PY{p}{,} \PY{l+m+mf}{0.9}\PY{p}{,} \PY{l+m+mf}{0.95}\PY{p}{,} \PY{l+m+mf}{1.0}\PY{p}{,} \PY{l+m+mf}{1.1}\PY{p}{,} \PY{l+m+mf}{1.2}\PY{p}{,} \PY{l+m+mf}{1.3}\PY{p}{,} \PY{l+m+mf}{1.4}\PY{p}{,} \PY{l+m+mf}{1.5}\PY{p}{,} \PY{l+m+mf}{1.6}\PY{p}{,} \PY{l+m+mf}{1.7}\PY{p}{,} \PY{l+m+mf}{1.8}\PY{p}{,} \PY{l+m+mf}{1.9}\PY{p}{,} \PY{l+m+mf}{2.0}\PY{p}{,} \PY{l+m+mf}{2.1}\PY{p}{,} \PY{l+m+mf}{2.2}\PY{p}{,} \PY{l+m+mf}{2.3}\PY{p}{]}\PY{p}{)}
         \PY{c+c1}{\PYZsh{}\PYZsh{}\PYZsh{} array for different instances of the HF molecule}
         \PY{n}{molecules} \PY{o}{=}\PY{p}{[}\PY{p}{]}
         \PY{c+c1}{\PYZsh{}\PYZsh{}\PYZsh{} array for the different RHF energies for different HF bond\PYZhy{}lengths}
         \PY{n}{HF\PYZus{}E\PYZus{}array} \PY{o}{=} \PY{p}{[}\PY{p}{]}
         \PY{c+c1}{\PYZsh{}\PYZsh{}\PYZsh{} array for the different MP2 energies for different HF bond\PYZhy{}lengths}
         \PY{n}{MP2\PYZus{}E\PYZus{}array} \PY{o}{=} \PY{p}{[}\PY{p}{]}
         \PY{c+c1}{\PYZsh{}\PYZsh{}\PYZsh{} array for the different CCSD energies for different HF bond\PYZhy{}lengths}
         \PY{n}{CCSD\PYZus{}E\PYZus{}array} \PY{o}{=} \PY{p}{[}\PY{p}{]}
         
         \PY{c+c1}{\PYZsh{}\PYZsh{}\PYZsh{} loop over the different bond\PYZhy{}lengths, create different instances}
         \PY{c+c1}{\PYZsh{}\PYZsh{}\PYZsh{} of HF molecule}
         \PY{k}{for} \PY{n}{r} \PY{o+ow}{in} \PY{n}{r\PYZus{}array}\PY{p}{:}
             \PY{n}{molecule} \PY{o}{=} \PY{n}{psi4}\PY{o}{.}\PY{n}{geometry}\PY{p}{(}\PY{n}{mol\PYZus{}tmpl}\PY{o}{.}\PY{n}{replace}\PY{p}{(}\PY{l+s+s2}{\PYZdq{}}\PY{l+s+s2}{**R**}\PY{l+s+s2}{\PYZdq{}}\PY{p}{,} \PY{n+nb}{str}\PY{p}{(}\PY{n}{r}\PY{p}{)}\PY{p}{)}\PY{p}{)}
             \PY{n}{molecules}\PY{o}{.}\PY{n}{append}\PY{p}{(}\PY{n}{molecule}\PY{p}{)}
             
         \PY{c+c1}{\PYZsh{}\PYZsh{}\PYZsh{} loop over instances of molecules, compute the RHF, MP2, and CCSD}
         \PY{c+c1}{\PYZsh{}\PYZsh{}\PYZsh{} energies and store them in their respective arrays}
         \PY{k}{for} \PY{n}{mol} \PY{o+ow}{in} \PY{n}{molecules}\PY{p}{:}
             \PY{n}{energy} \PY{o}{=} \PY{n}{psi4}\PY{o}{.}\PY{n}{energy}\PY{p}{(}\PY{l+s+s2}{\PYZdq{}}\PY{l+s+s2}{SCF/cc\PYZhy{}pVTZ}\PY{l+s+s2}{\PYZdq{}}\PY{p}{,} \PY{n}{molecule}\PY{o}{=}\PY{n}{mol}\PY{p}{)}
             \PY{n}{HF\PYZus{}E\PYZus{}array}\PY{o}{.}\PY{n}{append}\PY{p}{(}\PY{n}{energy}\PY{p}{)}
             \PY{n}{energy} \PY{o}{=} \PY{n}{psi4}\PY{o}{.}\PY{n}{energy}\PY{p}{(}\PY{l+s+s2}{\PYZdq{}}\PY{l+s+s2}{MP2/cc\PYZhy{}pVTZ}\PY{l+s+s2}{\PYZdq{}}\PY{p}{,} \PY{n}{molecule}\PY{o}{=}\PY{n}{mol}\PY{p}{)}
             \PY{n}{MP2\PYZus{}E\PYZus{}array}\PY{o}{.}\PY{n}{append}\PY{p}{(}\PY{n}{energy}\PY{p}{)}
             \PY{n}{energy} \PY{o}{=} \PY{n}{psi4}\PY{o}{.}\PY{n}{energy}\PY{p}{(}\PY{l+s+s2}{\PYZdq{}}\PY{l+s+s2}{CCSD/cc\PYZhy{}pVTZ}\PY{l+s+s2}{\PYZdq{}}\PY{p}{,}\PY{n}{molecule}\PY{o}{=}\PY{n}{mol}\PY{p}{)}
             \PY{n}{CCSD\PYZus{}E\PYZus{}array}\PY{o}{.}\PY{n}{append}\PY{p}{(}\PY{n}{energy}\PY{p}{)}
         
         \PY{c+c1}{\PYZsh{}\PYZsh{}\PYZsh{} Plot the 3 different PES}
         \PY{n}{plt}\PY{o}{.}\PY{n}{plot}\PY{p}{(}\PY{n}{r\PYZus{}array}\PY{p}{,}\PY{n}{HF\PYZus{}E\PYZus{}array}\PY{p}{,}\PY{l+s+s1}{\PYZsq{}}\PY{l+s+s1}{r*}\PY{l+s+s1}{\PYZsq{}}\PY{p}{,} \PY{n}{label}\PY{o}{=}\PY{l+s+s1}{\PYZsq{}}\PY{l+s+s1}{RHF}\PY{l+s+s1}{\PYZsq{}}\PY{p}{)}
         \PY{n}{plt}\PY{o}{.}\PY{n}{plot}\PY{p}{(}\PY{n}{r\PYZus{}array}\PY{p}{,}\PY{n}{MP2\PYZus{}E\PYZus{}array}\PY{p}{,}\PY{l+s+s1}{\PYZsq{}}\PY{l+s+s1}{g*}\PY{l+s+s1}{\PYZsq{}}\PY{p}{,} \PY{n}{label}\PY{o}{=}\PY{l+s+s1}{\PYZsq{}}\PY{l+s+s1}{MP2}\PY{l+s+s1}{\PYZsq{}}\PY{p}{)}
         \PY{n}{plt}\PY{o}{.}\PY{n}{plot}\PY{p}{(}\PY{n}{r\PYZus{}array}\PY{p}{,}\PY{n}{CCSD\PYZus{}E\PYZus{}array}\PY{p}{,}\PY{l+s+s1}{\PYZsq{}}\PY{l+s+s1}{b*}\PY{l+s+s1}{\PYZsq{}}\PY{p}{,} \PY{n}{label}\PY{o}{=}\PY{l+s+s1}{\PYZsq{}}\PY{l+s+s1}{CCSD}\PY{l+s+s1}{\PYZsq{}}\PY{p}{)}
         \PY{n}{plt}\PY{o}{.}\PY{n}{legend}\PY{p}{(}\PY{p}{)}
\end{Verbatim}


\begin{Verbatim}[commandchars=\\\{\}]
{\color{outcolor}Out[{\color{outcolor}16}]:} <matplotlib.legend.Legend at 0xb2fd76710>
\end{Verbatim}
            
    \begin{center}
    \adjustimage{max size={0.9\linewidth}{0.9\paperheight}}{output_2_1.png}
    \end{center}
    { \hspace*{\fill} \\}
    
    Now that you have the raw data, we will interpolate this data using
cubic splines. This will permit us to estimate the potential energy at
any arbitrary separation between 0.5 and 2.3 Angstroms (roughly 1 and
4.3 a.u.) with fairly high confidence, and will also allow us to
estimate the force (the negative of the derivative of the PES with
respect to separation) at any separation between 1.0 and 4.3 a.u. since
the derivative of cubic splines are readily available.

    \begin{Verbatim}[commandchars=\\\{\}]
{\color{incolor}In [{\color{incolor}4}]:} \PY{c+c1}{\PYZsh{}\PYZsh{}\PYZsh{} use cubic spline interpolation}
        \PY{n}{order} \PY{o}{=} \PY{l+m+mi}{3}
        
        \PY{c+c1}{\PYZsh{}\PYZsh{}\PYZsh{} get separation vector in atomic units}
        \PY{n}{r\PYZus{}array\PYZus{}au} \PY{o}{=} \PY{l+m+mf}{1.89}\PY{o}{*}\PY{n}{r\PYZus{}array} 
        
        \PY{c+c1}{\PYZsh{}\PYZsh{}\PYZsh{} spline for RHF Energy}
        \PY{n}{RHF\PYZus{}E\PYZus{}Spline} \PY{o}{=} \PY{n}{InterpolatedUnivariateSpline}\PY{p}{(}\PY{n}{r\PYZus{}array\PYZus{}au}\PY{p}{,} \PY{n}{HF\PYZus{}E\PYZus{}array}\PY{p}{,} \PY{n}{k}\PY{o}{=}\PY{n}{order}\PY{p}{)}
        
        \PY{c+c1}{\PYZsh{}\PYZsh{}\PYZsh{} spline for MP2 Energy}
        \PY{n}{MP2\PYZus{}E\PYZus{}Spline} \PY{o}{=} \PY{n}{InterpolatedUnivariateSpline}\PY{p}{(}\PY{n}{r\PYZus{}array\PYZus{}au}\PY{p}{,} \PY{n}{MP2\PYZus{}E\PYZus{}array}\PY{p}{,} \PY{n}{k}\PY{o}{=}\PY{n}{order}\PY{p}{)}
        
        \PY{c+c1}{\PYZsh{}\PYZsh{}\PYZsh{} spline for CCSD Energy}
        \PY{n}{CCSD\PYZus{}E\PYZus{}Spline} \PY{o}{=} \PY{n}{InterpolatedUnivariateSpline}\PY{p}{(}\PY{n}{r\PYZus{}array\PYZus{}au}\PY{p}{,} \PY{n}{CCSD\PYZus{}E\PYZus{}array}\PY{p}{,} \PY{n}{k}\PY{o}{=}\PY{n}{order}\PY{p}{)}
        
        
        \PY{c+c1}{\PYZsh{}\PYZsh{}\PYZsh{} form a much finer grid}
        \PY{n}{r\PYZus{}fine} \PY{o}{=} \PY{n}{np}\PY{o}{.}\PY{n}{linspace}\PY{p}{(}\PY{l+m+mf}{0.5}\PY{o}{/}\PY{l+m+mf}{0.529}\PY{p}{,}\PY{l+m+mf}{2.3}\PY{o}{/}\PY{l+m+mf}{0.529}\PY{p}{,}\PY{l+m+mi}{200}\PY{p}{)}
        
        \PY{c+c1}{\PYZsh{}\PYZsh{}\PYZsh{} compute the interpolated/extrapolated values for RHF Energy on this grid}
        \PY{n}{RHF\PYZus{}E\PYZus{}fine} \PY{o}{=} \PY{n}{RHF\PYZus{}E\PYZus{}Spline}\PY{p}{(}\PY{n}{r\PYZus{}fine}\PY{p}{)}
        
        \PY{c+c1}{\PYZsh{}\PYZsh{}\PYZsh{} compute the interpolated/extrapolated values for RHF Energy on this grid}
        \PY{n}{MP2\PYZus{}E\PYZus{}fine} \PY{o}{=} \PY{n}{MP2\PYZus{}E\PYZus{}Spline}\PY{p}{(}\PY{n}{r\PYZus{}fine}\PY{p}{)}
        
        \PY{c+c1}{\PYZsh{}\PYZsh{}\PYZsh{} compute the interpolated/extrapolated values for RHF Energy on this grid}
        \PY{n}{CCSD\PYZus{}E\PYZus{}fine} \PY{o}{=} \PY{n}{CCSD\PYZus{}E\PYZus{}Spline}\PY{p}{(}\PY{n}{r\PYZus{}fine}\PY{p}{)}
        
        
        \PY{c+c1}{\PYZsh{}\PYZsh{}\PYZsh{} plot the interpolated data}
        \PY{n}{plt}\PY{o}{.}\PY{n}{plot}\PY{p}{(}\PY{n}{r\PYZus{}fine}\PY{p}{,} \PY{n}{RHF\PYZus{}E\PYZus{}fine}\PY{p}{,} \PY{l+s+s1}{\PYZsq{}}\PY{l+s+s1}{red}\PY{l+s+s1}{\PYZsq{}}\PY{p}{,} \PY{n}{r\PYZus{}array\PYZus{}au}\PY{p}{,} \PY{n}{HF\PYZus{}E\PYZus{}array}\PY{p}{,} \PY{l+s+s1}{\PYZsq{}}\PY{l+s+s1}{r*}\PY{l+s+s1}{\PYZsq{}}\PY{p}{)}
        \PY{n}{plt}\PY{o}{.}\PY{n}{plot}\PY{p}{(}\PY{n}{r\PYZus{}fine}\PY{p}{,} \PY{n}{MP2\PYZus{}E\PYZus{}fine}\PY{p}{,} \PY{l+s+s1}{\PYZsq{}}\PY{l+s+s1}{green}\PY{l+s+s1}{\PYZsq{}}\PY{p}{,} \PY{n}{r\PYZus{}array\PYZus{}au}\PY{p}{,} \PY{n}{MP2\PYZus{}E\PYZus{}array}\PY{p}{,} \PY{l+s+s1}{\PYZsq{}}\PY{l+s+s1}{g*}\PY{l+s+s1}{\PYZsq{}}\PY{p}{)}
        \PY{n}{plt}\PY{o}{.}\PY{n}{plot}\PY{p}{(}\PY{n}{r\PYZus{}fine}\PY{p}{,} \PY{n}{CCSD\PYZus{}E\PYZus{}fine}\PY{p}{,} \PY{l+s+s1}{\PYZsq{}}\PY{l+s+s1}{blue}\PY{l+s+s1}{\PYZsq{}}\PY{p}{,} \PY{n}{r\PYZus{}array\PYZus{}au}\PY{p}{,} \PY{n}{CCSD\PYZus{}E\PYZus{}array}\PY{p}{,} \PY{l+s+s1}{\PYZsq{}}\PY{l+s+s1}{b*}\PY{l+s+s1}{\PYZsq{}}\PY{p}{)}
        \PY{n}{plt}\PY{o}{.}\PY{n}{show}\PY{p}{(}\PY{p}{)}
\end{Verbatim}


    \begin{center}
    \adjustimage{max size={0.9\linewidth}{0.9\paperheight}}{output_4_0.png}
    \end{center}
    { \hspace*{\fill} \\}
    
    We can estimate the equilibrium bond length by finding the separation at
which the potential is minimum; note this would also be the position
that the force goes to zero:

\begin{equation}
\frac{d}{dr} V(r_{eq}) = -F(r_{eq}) = 0.
\end{equation}

First we will compute the forces at each level of theory:

    \begin{Verbatim}[commandchars=\\\{\}]
{\color{incolor}In [{\color{incolor}5}]:} \PY{c+c1}{\PYZsh{}\PYZsh{}\PYZsh{} take the derivative of the potential to get the negative of the force from RHF}
        \PY{n}{RHF\PYZus{}Force} \PY{o}{=} \PY{n}{RHF\PYZus{}E\PYZus{}Spline}\PY{o}{.}\PY{n}{derivative}\PY{p}{(}\PY{p}{)} 
        
        \PY{c+c1}{\PYZsh{}\PYZsh{}\PYZsh{} negative of the force from MP2}
        \PY{n}{MP2\PYZus{}Force} \PY{o}{=} \PY{n}{MP2\PYZus{}E\PYZus{}Spline}\PY{o}{.}\PY{n}{derivative}\PY{p}{(}\PY{p}{)}
        
        \PY{c+c1}{\PYZsh{}\PYZsh{}\PYZsh{} negative of the force from CCSD}
        \PY{n}{CCSD\PYZus{}Force} \PY{o}{=} \PY{n}{CCSD\PYZus{}E\PYZus{}Spline}\PY{o}{.}\PY{n}{derivative}\PY{p}{(}\PY{p}{)}
        
        \PY{c+c1}{\PYZsh{}\PYZsh{}\PYZsh{} let\PYZsq{}s plot the forces for each level of theory!}
        
        \PY{c+c1}{\PYZsh{}\PYZsh{}\PYZsh{} plot the forces... note we need to multiply by \PYZhy{}1 since the spline}
        \PY{c+c1}{\PYZsh{}\PYZsh{}\PYZsh{} derivative gave us the negative of the force!}
        \PY{n}{plt}\PY{o}{.}\PY{n}{plot}\PY{p}{(}\PY{n}{r\PYZus{}fine}\PY{p}{,} \PY{o}{\PYZhy{}}\PY{l+m+mi}{1}\PY{o}{*}\PY{n}{RHF\PYZus{}Force}\PY{p}{(}\PY{n}{r\PYZus{}fine}\PY{p}{)}\PY{p}{,} \PY{l+s+s1}{\PYZsq{}}\PY{l+s+s1}{red}\PY{l+s+s1}{\PYZsq{}}\PY{p}{)}
        \PY{n}{plt}\PY{o}{.}\PY{n}{plot}\PY{p}{(}\PY{n}{r\PYZus{}fine}\PY{p}{,} \PY{o}{\PYZhy{}}\PY{l+m+mi}{1}\PY{o}{*}\PY{n}{MP2\PYZus{}Force}\PY{p}{(}\PY{n}{r\PYZus{}fine}\PY{p}{)}\PY{p}{,} \PY{l+s+s1}{\PYZsq{}}\PY{l+s+s1}{green}\PY{l+s+s1}{\PYZsq{}}\PY{p}{)}
        \PY{n}{plt}\PY{o}{.}\PY{n}{plot}\PY{p}{(}\PY{n}{r\PYZus{}fine}\PY{p}{,} \PY{o}{\PYZhy{}}\PY{l+m+mi}{1}\PY{o}{*}\PY{n}{CCSD\PYZus{}Force}\PY{p}{(}\PY{n}{r\PYZus{}fine}\PY{p}{)}\PY{p}{,} \PY{l+s+s1}{\PYZsq{}}\PY{l+s+s1}{blue}\PY{l+s+s1}{\PYZsq{}}\PY{p}{)}
        \PY{n}{plt}\PY{o}{.}\PY{n}{show}\PY{p}{(}\PY{p}{)}
\end{Verbatim}


    \begin{center}
    \adjustimage{max size={0.9\linewidth}{0.9\paperheight}}{output_6_0.png}
    \end{center}
    { \hspace*{\fill} \\}
    
    Next we will find where the minimum of the potential energy surfaces are
and use that to find the equilibrium bond length:

    \begin{Verbatim}[commandchars=\\\{\}]
{\color{incolor}In [{\color{incolor}6}]:} \PY{c+c1}{\PYZsh{}\PYZsh{}\PYZsh{} Find Equilibrium Bond\PYZhy{}Lengths for each level of theory}
        \PY{n}{RHF\PYZus{}Req\PYZus{}idx} \PY{o}{=} \PY{n}{np}\PY{o}{.}\PY{n}{argmin}\PY{p}{(}\PY{n}{RHF\PYZus{}E\PYZus{}fine}\PY{p}{)}
        \PY{n}{MP2\PYZus{}Req\PYZus{}idx} \PY{o}{=} \PY{n}{np}\PY{o}{.}\PY{n}{argmin}\PY{p}{(}\PY{n}{MP2\PYZus{}E\PYZus{}fine}\PY{p}{)}
        \PY{n}{CCSD\PYZus{}Req\PYZus{}idx} \PY{o}{=} \PY{n}{np}\PY{o}{.}\PY{n}{argmin}\PY{p}{(}\PY{n}{CCSD\PYZus{}E\PYZus{}fine}\PY{p}{)}
        
        \PY{c+c1}{\PYZsh{}\PYZsh{}\PYZsh{} find the value of the separation corresponding to that index}
        \PY{n}{RHF\PYZus{}Req} \PY{o}{=} \PY{n}{r\PYZus{}fine}\PY{p}{[}\PY{n}{RHF\PYZus{}Req\PYZus{}idx}\PY{p}{]}
        \PY{n}{MP2\PYZus{}Req} \PY{o}{=} \PY{n}{r\PYZus{}fine}\PY{p}{[}\PY{n}{MP2\PYZus{}Req\PYZus{}idx}\PY{p}{]}
        \PY{n}{CCSD\PYZus{}Req} \PY{o}{=} \PY{n}{r\PYZus{}fine}\PY{p}{[}\PY{n}{CCSD\PYZus{}Req\PYZus{}idx}\PY{p}{]}
        
        \PY{c+c1}{\PYZsh{}\PYZsh{}\PYZsh{} print equilibrium bond\PYZhy{}lengths at each level of theory!}
        \PY{n+nb}{print}\PY{p}{(}\PY{l+s+s2}{\PYZdq{}}\PY{l+s+s2}{ Equilibrium bond length at RHF/cc\PYZhy{}pVDZ level is }\PY{l+s+s2}{\PYZdq{}}\PY{p}{,}\PY{n}{RHF\PYZus{}Req}\PY{p}{,} \PY{l+s+s2}{\PYZdq{}}\PY{l+s+s2}{atomic units}\PY{l+s+s2}{\PYZdq{}}\PY{p}{)}
        \PY{n+nb}{print}\PY{p}{(}\PY{l+s+s2}{\PYZdq{}}\PY{l+s+s2}{ Equilibrium bond length at MP2/cc\PYZhy{}pVDZ level is }\PY{l+s+s2}{\PYZdq{}}\PY{p}{,}\PY{n}{MP2\PYZus{}Req}\PY{p}{,} \PY{l+s+s2}{\PYZdq{}}\PY{l+s+s2}{atomic units}\PY{l+s+s2}{\PYZdq{}}\PY{p}{)}
        \PY{n+nb}{print}\PY{p}{(}\PY{l+s+s2}{\PYZdq{}}\PY{l+s+s2}{ Equilibrium bond lengthat CCSD/cc\PYZhy{}pVDZ level is }\PY{l+s+s2}{\PYZdq{}}\PY{p}{,}\PY{n}{CCSD\PYZus{}Req}\PY{p}{,} \PY{l+s+s2}{\PYZdq{}}\PY{l+s+s2}{atomic units}\PY{l+s+s2}{\PYZdq{}}\PY{p}{)}
\end{Verbatim}


    \begin{Verbatim}[commandchars=\\\{\}]
 Equilibrium bond length at RHF/cc-pVDZ level is  1.6975235344966797 atomic units
 Equilibrium bond length at MP2/cc-pVDZ level is  1.7317209867864842 atomic units
 Equilibrium bond lengthat CCSD/cc-pVDZ level is  1.7317209867864842 atomic units

    \end{Verbatim}

    At this point, take a moment to compare your equilibrium bond length
with other teams who have used different levels of theory to compute
their potential energy surfaces. Which equilibrium bond length should be
most trustworthy?

You might have learned that the Harmonic Oscillator potential, which is
a reasonable model for the vibrational motion of diatomic atomcs near
their equilibrium bond length, is given by

\begin{equation}
V(r) = \frac{1}{2} k r^2
\end{equation}

and that the vibrational frequency of the molecule within the Harmonic
oscillator model is given by

\begin{equation}
\nu = \frac{1}{2\pi}\sqrt{\frac{k}{\mu}}
\end{equation}

where \(\mu\) is the reduced mass of the molecule and \(k\) is known as
the force constant.\\
We can estimate the force constant as

\begin{equation}
k = \frac{d^2}{dr^2} V(r_{eq}).
\end{equation}

Let's go ahead and get the force constants at each level of theory and
estimate the potential energy within the Harmonic approximation!

    \begin{Verbatim}[commandchars=\\\{\}]
{\color{incolor}In [{\color{incolor}7}]:} \PY{c+c1}{\PYZsh{}\PYZsh{}\PYZsh{} get second derivative of potential energy curve... recall that we fit a spline to}
        \PY{c+c1}{\PYZsh{}\PYZsh{}\PYZsh{} to the first derivative already and called that spline function X\PYZus{}Force, where}
        \PY{c+c1}{\PYZsh{}\PYZsh{}\PYZsh{} X is either RHF, MP2, or CCSD}
        
        \PY{n}{RHF\PYZus{}Curvature} \PY{o}{=} \PY{n}{RHF\PYZus{}Force}\PY{o}{.}\PY{n}{derivative}\PY{p}{(}\PY{p}{)}
        \PY{n}{MP2\PYZus{}Curvature} \PY{o}{=} \PY{n}{MP2\PYZus{}Force}\PY{o}{.}\PY{n}{derivative}\PY{p}{(}\PY{p}{)}
        \PY{n}{CCSD\PYZus{}Curvature} \PY{o}{=} \PY{n}{CCSD\PYZus{}Force}\PY{o}{.}\PY{n}{derivative}\PY{p}{(}\PY{p}{)}
        
        \PY{c+c1}{\PYZsh{}\PYZsh{}\PYZsh{} evaluate the second derivative at r\PYZus{}eq to get k}
        \PY{n}{RHF\PYZus{}k} \PY{o}{=} \PY{n}{RHF\PYZus{}Curvature}\PY{p}{(}\PY{n}{RHF\PYZus{}Req}\PY{p}{)}
        \PY{n}{MP2\PYZus{}k} \PY{o}{=} \PY{n}{MP2\PYZus{}Curvature}\PY{p}{(}\PY{n}{MP2\PYZus{}Req}\PY{p}{)}
        \PY{n}{CCSD\PYZus{}k} \PY{o}{=} \PY{n}{CCSD\PYZus{}Curvature}\PY{p}{(}\PY{n}{CCSD\PYZus{}Req}\PY{p}{)}
        
        \PY{c+c1}{\PYZsh{}\PYZsh{}\PYZsh{} Print force constants for each level of theory!}
        \PY{n+nb}{print}\PY{p}{(}\PY{l+s+s2}{\PYZdq{}}\PY{l+s+s2}{Hartree\PYZhy{}Fock force constant is }\PY{l+s+s2}{\PYZdq{}}\PY{p}{,}\PY{n}{RHF\PYZus{}k}\PY{p}{,}\PY{l+s+s2}{\PYZdq{}}\PY{l+s+s2}{ atomic units}\PY{l+s+s2}{\PYZdq{}}\PY{p}{)}
        \PY{n+nb}{print}\PY{p}{(}\PY{l+s+s2}{\PYZdq{}}\PY{l+s+s2}{MP2 force constant is }\PY{l+s+s2}{\PYZdq{}}\PY{p}{,}\PY{n}{MP2\PYZus{}k}\PY{p}{,}\PY{l+s+s2}{\PYZdq{}}\PY{l+s+s2}{ atomic units}\PY{l+s+s2}{\PYZdq{}}\PY{p}{)}
        \PY{n+nb}{print}\PY{p}{(}\PY{l+s+s2}{\PYZdq{}}\PY{l+s+s2}{CCSD force constant is }\PY{l+s+s2}{\PYZdq{}}\PY{p}{,}\PY{n}{CCSD\PYZus{}k}\PY{p}{,}\PY{l+s+s2}{\PYZdq{}}\PY{l+s+s2}{ atomic units}\PY{l+s+s2}{\PYZdq{}}\PY{p}{)}
        
        
        \PY{c+c1}{\PYZsh{}\PYZsh{}\PYZsh{} define harmonic potential for each level of theory}
        \PY{n}{RHF\PYZus{}Harm\PYZus{}Pot} \PY{o}{=} \PY{n}{RHF\PYZus{}k}\PY{o}{*}\PY{p}{(}\PY{n}{r\PYZus{}fine}\PY{o}{\PYZhy{}}\PY{n}{RHF\PYZus{}Req}\PY{p}{)}\PY{o}{*}\PY{o}{*}\PY{l+m+mi}{2} \PY{o}{+} \PY{n}{RHF\PYZus{}E\PYZus{}Spline}\PY{p}{(}\PY{n}{RHF\PYZus{}Req}\PY{p}{)}
        \PY{n}{MP2\PYZus{}Harm\PYZus{}Pot} \PY{o}{=} \PY{n}{MP2\PYZus{}k}\PY{o}{*}\PY{p}{(}\PY{n}{r\PYZus{}fine}\PY{o}{\PYZhy{}}\PY{n}{MP2\PYZus{}Req}\PY{p}{)}\PY{o}{*}\PY{o}{*}\PY{l+m+mi}{2} \PY{o}{+} \PY{n}{MP2\PYZus{}E\PYZus{}Spline}\PY{p}{(}\PY{n}{MP2\PYZus{}Req}\PY{p}{)}
        \PY{n}{CCSD\PYZus{}Harm\PYZus{}Pot} \PY{o}{=} \PY{n}{CCSD\PYZus{}k}\PY{o}{*}\PY{p}{(}\PY{n}{r\PYZus{}fine}\PY{o}{\PYZhy{}}\PY{n}{CCSD\PYZus{}Req}\PY{p}{)}\PY{o}{*}\PY{o}{*}\PY{l+m+mi}{2} \PY{o}{+} \PY{n}{CCSD\PYZus{}E\PYZus{}Spline}\PY{p}{(}\PY{n}{CCSD\PYZus{}Req}\PY{p}{)}
        
        
        \PY{c+c1}{\PYZsh{}\PYZsh{}\PYZsh{} plot!}
        \PY{n}{plt}\PY{o}{.}\PY{n}{plot}\PY{p}{(}\PY{n}{r\PYZus{}fine}\PY{p}{,} \PY{n}{RHF\PYZus{}Harm\PYZus{}Pot}\PY{p}{,} \PY{l+s+s1}{\PYZsq{}}\PY{l+s+s1}{red}\PY{l+s+s1}{\PYZsq{}}\PY{p}{)}
        \PY{n}{plt}\PY{o}{.}\PY{n}{plot}\PY{p}{(}\PY{n}{r\PYZus{}fine}\PY{p}{,} \PY{n}{MP2\PYZus{}Harm\PYZus{}Pot}\PY{p}{,} \PY{l+s+s1}{\PYZsq{}}\PY{l+s+s1}{green}\PY{l+s+s1}{\PYZsq{}}\PY{p}{)}
        \PY{n}{plt}\PY{o}{.}\PY{n}{plot}\PY{p}{(}\PY{n}{r\PYZus{}fine}\PY{p}{,} \PY{n}{CCSD\PYZus{}Harm\PYZus{}Pot}\PY{p}{,} \PY{l+s+s1}{\PYZsq{}}\PY{l+s+s1}{blue}\PY{l+s+s1}{\PYZsq{}}\PY{p}{)}
        \PY{n}{plt}\PY{o}{.}\PY{n}{show}\PY{p}{(}\PY{p}{)}
\end{Verbatim}


    \begin{Verbatim}[commandchars=\\\{\}]
Hartree-Fock force constant is  0.7201717151234356  atomic units
MP2 force constant is  0.642303342116713  atomic units
CCSD force constant is  0.6393333547805894  atomic units

    \end{Verbatim}

    \begin{center}
    \adjustimage{max size={0.9\linewidth}{0.9\paperheight}}{output_10_1.png}
    \end{center}
    { \hspace*{\fill} \\}
    
    And we can actually estimate the fundamental vibrational frequency of
the molecule within this model using the force constant and the reduced
mass of the molecule.

\subsubsection{Question 1: What is the reduced mass of the HF molecule
in atomic
units?}\label{question-1-what-is-the-reduced-mass-of-the-hf-molecule-in-atomic-units}

\subsubsection{Question 2: Use your spline fit to the PES of HF to
estimate the vibrational frequency of HF. Express your number in atomic
units and also convert to a common spectroscopic unit system of your
choosing (wavenumbers, nm, microns, Hertz, THz are all acceptable
choices).}\label{question-2-use-your-spline-fit-to-the-pes-of-hf-to-estimate-the-vibrational-frequency-of-hf.-express-your-number-in-atomic-units-and-also-convert-to-a-common-spectroscopic-unit-system-of-your-choosing-wavenumbers-nm-microns-hertz-thz-are-all-acceptable-choices.}

    \begin{Verbatim}[commandchars=\\\{\}]
{\color{incolor}In [{\color{incolor}8}]:} \PY{c+c1}{\PYZsh{}HE = InterpolatedUnivariateSpline(r\PYZus{}fine, harm\PYZus{}pot, k=order)}
        \PY{c+c1}{\PYZsh{}HF = \PYZhy{}HE.derivative()}
        \PY{c+c1}{\PYZsh{}\PYZsh{}\PYZsh{} define reduced mass of HF as m\PYZus{}H * m\PYZus{}H /(m\PYZus{}F + m\PYZus{}H) where mass is in atomic units (electron mass = 1)}
        \PY{n}{m\PYZus{}F} \PY{o}{=} \PY{l+m+mf}{34883.}
        \PY{n}{m\PYZus{}H} \PY{o}{=} \PY{l+m+mf}{1836.}
        \PY{n}{mu} \PY{o}{=} \PY{p}{(}\PY{n}{m\PYZus{}F} \PY{o}{*} \PY{n}{m\PYZus{}H}\PY{p}{)}\PY{o}{/}\PY{p}{(}\PY{n}{m\PYZus{}F} \PY{o}{+} \PY{n}{m\PYZus{}H}\PY{p}{)}
        
        \PY{c+c1}{\PYZsh{}\PYZsh{}\PYZsh{} compute the fundamental frequency at each level of theory}
        \PY{n}{RHF\PYZus{}nu} \PY{o}{=} \PY{l+m+mi}{1}\PY{o}{/}\PY{p}{(}\PY{n}{np}\PY{o}{.}\PY{n}{pi}\PY{o}{*}\PY{l+m+mi}{2}\PY{p}{)} \PY{o}{*} \PY{n}{np}\PY{o}{.}\PY{n}{sqrt}\PY{p}{(}\PY{n}{RHF\PYZus{}k}\PY{o}{/}\PY{n}{mu}\PY{p}{)}
        \PY{n}{MP2\PYZus{}nu} \PY{o}{=} \PY{l+m+mi}{1}\PY{o}{/}\PY{p}{(}\PY{n}{np}\PY{o}{.}\PY{n}{pi}\PY{o}{*}\PY{l+m+mi}{2}\PY{p}{)} \PY{o}{*} \PY{n}{np}\PY{o}{.}\PY{n}{sqrt}\PY{p}{(}\PY{n}{MP2\PYZus{}k}\PY{o}{/}\PY{n}{mu}\PY{p}{)}
        \PY{n}{CCSD\PYZus{}nu} \PY{o}{=} \PY{l+m+mi}{1}\PY{o}{/}\PY{p}{(}\PY{n}{np}\PY{o}{.}\PY{n}{pi}\PY{o}{*}\PY{l+m+mi}{2}\PY{p}{)} \PY{o}{*} \PY{n}{np}\PY{o}{.}\PY{n}{sqrt}\PY{p}{(}\PY{n}{CCSD\PYZus{}k}\PY{o}{/}\PY{n}{mu}\PY{p}{)}
        
        \PY{c+c1}{\PYZsh{}\PYZsh{}\PYZsh{} print the values in atomic units!}
        \PY{n+nb}{print}\PY{p}{(}\PY{l+s+s2}{\PYZdq{}}\PY{l+s+s2}{Vibrational frequency of HF at the RHF/cc\PYZhy{}pVDZ level is }\PY{l+s+s2}{\PYZdq{}}\PY{p}{,}\PY{n}{RHF\PYZus{}nu}\PY{p}{,}\PY{l+s+s2}{\PYZdq{}}\PY{l+s+s2}{ atomic units}\PY{l+s+s2}{\PYZdq{}}\PY{p}{)}
        \PY{n+nb}{print}\PY{p}{(}\PY{l+s+s2}{\PYZdq{}}\PY{l+s+s2}{Vibrational frequency of HF at the MP2/cc\PYZhy{}pVDZ level is }\PY{l+s+s2}{\PYZdq{}}\PY{p}{,}\PY{n}{MP2\PYZus{}nu}\PY{p}{,}\PY{l+s+s2}{\PYZdq{}}\PY{l+s+s2}{ atomic units}\PY{l+s+s2}{\PYZdq{}}\PY{p}{)}
        \PY{n+nb}{print}\PY{p}{(}\PY{l+s+s2}{\PYZdq{}}\PY{l+s+s2}{Vibrational frequency of HF at the CCSD/cc\PYZhy{}pVDZ level is }\PY{l+s+s2}{\PYZdq{}}\PY{p}{,}\PY{n}{CCSD\PYZus{}nu}\PY{p}{,}\PY{l+s+s2}{\PYZdq{}}\PY{l+s+s2}{ atomic units}\PY{l+s+s2}{\PYZdq{}}\PY{p}{)}
\end{Verbatim}


    \begin{Verbatim}[commandchars=\\\{\}]
Vibrational frequency of HF at the RHF/cc-pVDZ level is  0.003234002356333305  atomic units
Vibrational frequency of HF at the MP2/cc-pVDZ level is  0.0030541642741115874  atomic units
Vibrational frequency of HF at the CCSD/cc-pVDZ level is  0.0030470949201703997  atomic units

    \end{Verbatim}

    \subsubsection{Question 3: How does these vibrational frequencies
compare to the experimental vibrational frequency of
HF?}\label{question-3-how-does-these-vibrational-frequencies-compare-to-the-experimental-vibrational-frequency-of-hf}

Next, we want to actually simulate the dynamics of the HF molecule on
this \emph{ab} \emph{initio} potential energy surface. To do so, we need
to solve Newton's equations of motion subject to some initial condition
for the position (separation) and momentum (in a relative sense) of the
particles. Newton's equations can be written

\begin{equation}
F(r) = \mu \frac{d^2}{dr^2}
\end{equation}

where \(\mu\) is the reduced mass in atomic units and \(F(r)\) is the
Force vs separation in atomic units that was determined previously.

\subsubsection{Question 4: What will be the accelation of the bond
stretch when H is separated by F by 3 atomic units? You can express your
acceleration in atomic units,
also.}\label{question-4-what-will-be-the-accelation-of-the-bond-stretch-when-h-is-separated-by-f-by-3-atomic-units-you-can-express-your-acceleration-in-atomic-units-also.}

    If the acceleration, position, and velocity of the bond stretch
coordinate are known at some instant in time \(t_i\), then the position
and velocity can be estimated at some later time
\(t_{i+1} = t_i + \Delta t\):

\begin{equation}
r(t_i + \Delta t) = r(t_i) + v(t_i)\Delta t + \frac{1}{2}a(t_i)\Delta t^2
\end{equation}

and

\begin{equation}
v(t_i + \Delta t) = v(t_i) + \frac{1}{2} \left(a(t_i) + a(t_i + \Delta t)  \right) \Delta t.
\end{equation}

This prescription for updating the velocities and positions is known as
the Velocity-Verlet algorithm.\\
Note that we need to perform 2 force evaluations per Velocity-Verlet
iteration: one corresponding to position \(r(t_i)\) to update the
position, and then a second time at the updated position
\(r(t_i + \Delta t)\) to complete the velocity update.

    \begin{Verbatim}[commandchars=\\\{\}]
{\color{incolor}In [{\color{incolor}9}]:} \PY{k}{def} \PY{n+nf}{Velocity\PYZus{}Verlet}\PY{p}{(}\PY{n}{r\PYZus{}curr}\PY{p}{,} \PY{n}{v\PYZus{}curr}\PY{p}{,} \PY{n}{mu}\PY{p}{,} \PY{n}{f\PYZus{}interp}\PY{p}{,} \PY{n}{dt}\PY{p}{)}\PY{p}{:}
            \PY{c+c1}{\PYZsh{}\PYZsh{}\PYZsh{} get acceleration at current time}
            \PY{n}{a\PYZus{}curr} \PY{o}{=} \PY{o}{\PYZhy{}}\PY{l+m+mi}{1}\PY{o}{*}\PY{n}{f\PYZus{}interp}\PY{p}{(}\PY{n}{r\PYZus{}curr}\PY{p}{)}\PY{o}{/}\PY{n}{mu}
            
            \PY{c+c1}{\PYZsh{}\PYZsh{}\PYZsh{} use current acceleration and velocity to update position}
            \PY{n}{r\PYZus{}fut} \PY{o}{=} \PY{n}{r\PYZus{}curr} \PY{o}{+} \PY{n}{v\PYZus{}curr} \PY{o}{*} \PY{n}{dt} \PY{o}{+} \PY{l+m+mf}{0.5} \PY{o}{*} \PY{n}{a\PYZus{}curr} \PY{o}{*} \PY{n}{dt}\PY{o}{*}\PY{o}{*}\PY{l+m+mi}{2}
            
            \PY{c+c1}{\PYZsh{}\PYZsh{}\PYZsh{} use r\PYZus{}fut to get future acceleration a\PYZus{}fut}
            \PY{n}{a\PYZus{}fut} \PY{o}{=} \PY{o}{\PYZhy{}}\PY{l+m+mi}{1}\PY{o}{*}\PY{n}{f\PYZus{}interp}\PY{p}{(}\PY{n}{r\PYZus{}fut}\PY{p}{)}\PY{o}{/}\PY{n}{mu}
            \PY{c+c1}{\PYZsh{}\PYZsh{}\PYZsh{} use current and future acceleration to get future velocity v\PYZus{}fut}
            \PY{n}{v\PYZus{}fut} \PY{o}{=} \PY{n}{v\PYZus{}curr} \PY{o}{+} \PY{l+m+mf}{0.5}\PY{o}{*}\PY{p}{(}\PY{n}{a\PYZus{}curr} \PY{o}{+} \PY{n}{a\PYZus{}fut}\PY{p}{)} \PY{o}{*} \PY{n}{dt}
            
            \PY{n}{result} \PY{o}{=} \PY{p}{[}\PY{n}{r\PYZus{}fut}\PY{p}{,} \PY{n}{v\PYZus{}fut}\PY{p}{]}
            
            \PY{k}{return} \PY{n}{result}
\end{Verbatim}


    To be able to define the very first update, an initial position and
velocity must be specified. Typically, these are chosen at random from a
sensible range of values.

\subsubsection{Question 5: What makes a "sensible range of values" for
position and
velocity?}\label{question-5-what-makes-a-sensible-range-of-values-for-position-and-velocity}

In this case, we will initialize the position to be a random number
between 1.0 and 4.0; for the velocity, we will use the fact that we can
estimate the expectation value of kinetic energy for a very similar
system (the Harmonic oscillator) in the ground state as follows:

\begin{equation}
\langle T \rangle = \frac{1}{2} E_g,
\end{equation}

where \(E_g\) is the ground state of the Harmonic oscillator (this is
making use of the Virial theorem). We can easily find the ground state
energy in the Harmonic oscillator approximation of \(HF\) using our
frequency calculation described above as

\begin{equation}
E_g = \frac{1}{2} h \nu,
\end{equation}

which implies the kinetic energy expectation value is

\begin{equation}
\langle T \rangle = \frac{h}{8 \pi} \sqrt{\frac{k}{\mu}}.
\end{equation}

Since we can say classically that the kinetic energy is given by
\(T = \frac{1}{2}\mu v^2\), we can estimate the velocity of the bond
stretch as follows:

\begin{equation}
v = \sqrt{\frac{2 \langle T \rangle}{\mu}} = \sqrt{ \frac{\hbar \sqrt{\frac{k}{\mu}}}{2\mu}}
\end{equation}

where we have simplified using the fact that \(\hbar = \frac{h}{2\pi}\)
(\(\hbar\) has the value 1 in the atomic unit system we are using up to
this point!). We will assume that a reasonable range of velocities spans
plus or minus 3 times this "ground-state" velocity.

    \begin{Verbatim}[commandchars=\\\{\}]
{\color{incolor}In [{\color{incolor}20}]:} \PY{c+c1}{\PYZsh{}\PYZsh{}\PYZsh{} define \PYZdq{}ground\PYZhy{}state\PYZdq{} velocity for each level of theory}
         \PY{n}{v\PYZus{}RHF} \PY{o}{=} \PY{n}{np}\PY{o}{.}\PY{n}{sqrt}\PY{p}{(} \PY{n}{np}\PY{o}{.}\PY{n}{sqrt}\PY{p}{(}\PY{n}{RHF\PYZus{}k}\PY{o}{/}\PY{n}{mu}\PY{p}{)}\PY{o}{/}\PY{p}{(}\PY{l+m+mi}{2}\PY{o}{*}\PY{n}{mu}\PY{p}{)}\PY{p}{)}
         \PY{n}{v\PYZus{}MP2} \PY{o}{=} \PY{n}{np}\PY{o}{.}\PY{n}{sqrt}\PY{p}{(} \PY{n}{np}\PY{o}{.}\PY{n}{sqrt}\PY{p}{(}\PY{n}{MP2\PYZus{}k}\PY{o}{/}\PY{n}{mu}\PY{p}{)}\PY{o}{/}\PY{p}{(}\PY{l+m+mi}{2}\PY{o}{*}\PY{n}{mu}\PY{p}{)}\PY{p}{)}
         \PY{n}{v\PYZus{}CCSD} \PY{o}{=} \PY{n}{np}\PY{o}{.}\PY{n}{sqrt}\PY{p}{(} \PY{n}{np}\PY{o}{.}\PY{n}{sqrt}\PY{p}{(}\PY{n}{CCSD\PYZus{}k}\PY{o}{/}\PY{n}{mu}\PY{p}{)}\PY{o}{/}\PY{p}{(}\PY{l+m+mi}{2}\PY{o}{*}\PY{n}{mu}\PY{p}{)}\PY{p}{)}
         
         
         \PY{c+c1}{\PYZsh{}\PYZsh{}\PYZsh{} get random position and velocity for RHF HF within a reasonable range}
         \PY{c+c1}{\PYZsh{}r\PYZus{}init = np.random.uniform(0.75*RHF\PYZus{}Req,2*RHF\PYZus{}Req)}
         \PY{n}{r\PYZus{}init} \PY{o}{=} \PY{n}{RHF\PYZus{}Req}
         \PY{n}{v\PYZus{}init} \PY{o}{=} \PY{n}{np}\PY{o}{.}\PY{n}{random}\PY{o}{.}\PY{n}{uniform}\PY{p}{(}\PY{o}{\PYZhy{}}\PY{l+m+mi}{2}\PY{o}{*}\PY{n}{v\PYZus{}RHF}\PY{p}{,}\PY{l+m+mi}{2}\PY{o}{*}\PY{n}{v\PYZus{}RHF}\PY{p}{)}
         
         \PY{c+c1}{\PYZsh{}\PYZsh{}\PYZsh{} print initial position and velocity}
         \PY{n+nb}{print}\PY{p}{(}\PY{l+s+s2}{\PYZdq{}}\PY{l+s+s2}{Initial separation is }\PY{l+s+s2}{\PYZdq{}}\PY{p}{,}\PY{n}{r\PYZus{}init}\PY{p}{,} \PY{l+s+s2}{\PYZdq{}}\PY{l+s+s2}{atomic units}\PY{l+s+s2}{\PYZdq{}}\PY{p}{)}
         \PY{n+nb}{print}\PY{p}{(}\PY{l+s+s2}{\PYZdq{}}\PY{l+s+s2}{Initial velocity is   }\PY{l+s+s2}{\PYZdq{}}\PY{p}{,}\PY{n}{v\PYZus{}init}\PY{p}{,} \PY{l+s+s2}{\PYZdq{}}\PY{l+s+s2}{atomic units}\PY{l+s+s2}{\PYZdq{}}\PY{p}{)}
         
         
         \PY{c+c1}{\PYZsh{}\PYZsh{}\PYZsh{} get initial force on the particle based on its separation}
         \PY{n}{RHF\PYZus{}F\PYZus{}init} \PY{o}{=} \PY{o}{\PYZhy{}}\PY{l+m+mi}{1}\PY{o}{*}\PY{n}{RHF\PYZus{}Force}\PY{p}{(}\PY{n}{r\PYZus{}init}\PY{p}{)}
         \PY{n+nb}{print}\PY{p}{(}\PY{l+s+s2}{\PYZdq{}}\PY{l+s+s2}{Initial Force is }\PY{l+s+s2}{\PYZdq{}}\PY{p}{,} \PY{n}{RHF\PYZus{}F\PYZus{}init}\PY{p}{,} \PY{l+s+s2}{\PYZdq{}}\PY{l+s+s2}{atomic units}\PY{l+s+s2}{\PYZdq{}}\PY{p}{)}
\end{Verbatim}


    \begin{Verbatim}[commandchars=\\\{\}]
Initial separation is  1.6975235344966797 atomic units
Initial velocity is    -0.00472954841024506 atomic units
Initial Force is  -0.00032337562265599695 atomic units

    \end{Verbatim}

    \subsubsection{Validating Velocity-Verlet algorithm with the Harmonic
Oscillator}\label{validating-velocity-verlet-algorithm-with-the-harmonic-oscillator}

Newton's equation of motion can be solved analytically for the Harmonic
oscillator, and we can use this fact to validate our Velocity-Verlet
algorithm (which provides an \emph{approximate} solution to Newton's
equation of motion for arbitrary potentials). That is, the vibrational
motion of a diatomic subject to a Harmonic potential predicted by the
Velocity-Verlet algorithm should closely match the analytical solution.
If the diatomic is initially at its equilibrium bond length
(\(r(0) = r_{eq})\), then the bond-length at subsequent times can be
written exactly as

\begin{equation}
r(t) = \sqrt{\frac{\mu}{k}} v_0 {\rm sin}\left(\sqrt{\frac{k}{\mu}} t\right) + r_{eq}.
\end{equation}

    \begin{Verbatim}[commandchars=\\\{\}]
{\color{incolor}In [{\color{incolor}24}]:} \PY{k}{def} \PY{n+nf}{harmonic\PYZus{}position}\PY{p}{(}\PY{n}{k\PYZus{}val}\PY{p}{,} \PY{n}{mu\PYZus{}val}\PY{p}{,} \PY{n}{v0\PYZus{}val}\PY{p}{,} \PY{n}{r\PYZus{}eq\PYZus{}val}\PY{p}{,} \PY{n}{time}\PY{p}{)}\PY{p}{:}
             \PY{n}{f1} \PY{o}{=} \PY{n}{np}\PY{o}{.}\PY{n}{sqrt}\PY{p}{(}\PY{n}{mu\PYZus{}val}\PY{o}{/}\PY{n}{k\PYZus{}val}\PY{p}{)}
             \PY{n}{f2} \PY{o}{=} \PY{n}{np}\PY{o}{.}\PY{n}{sqrt}\PY{p}{(}\PY{n}{k\PYZus{}val}\PY{o}{/}\PY{n}{mu\PYZus{}val}\PY{p}{)}
             \PY{k}{return} \PY{n}{f1} \PY{o}{*} \PY{n}{v0\PYZus{}val} \PY{o}{*} \PY{n}{np}\PY{o}{.}\PY{n}{sin}\PY{p}{(} \PY{n}{f2} \PY{o}{*} \PY{n}{time} \PY{p}{)} \PY{o}{+} \PY{n}{r\PYZus{}eq\PYZus{}val}
         
         \PY{c+c1}{\PYZsh{}\PYZsh{}\PYZsh{} how many updates do you want to perform?}
         \PY{n}{N\PYZus{}updates} \PY{o}{=} \PY{l+m+mi}{50000}
         
         \PY{c+c1}{\PYZsh{}\PYZsh{}\PYZsh{} establish time\PYZhy{}step for integration to be 0.02 atomic units... this is about 0.0005 femtoseconds}
         \PY{c+c1}{\PYZsh{}\PYZsh{}\PYZsh{} so total time is 200000*0.02 atomic units of time which is \PYZti{}9.6e\PYZhy{}13 s, or 960 fs}
         \PY{n}{dt} \PY{o}{=} \PY{l+m+mf}{0.001}
         
         \PY{c+c1}{\PYZsh{}\PYZsh{}\PYZsh{} harmonic results}
         \PY{n}{hr\PYZus{}vs\PYZus{}t} \PY{o}{=} \PY{n}{np}\PY{o}{.}\PY{n}{zeros}\PY{p}{(}\PY{n}{N\PYZus{}updates}\PY{p}{)}
         \PY{n}{hv\PYZus{}vs\PYZus{}t} \PY{o}{=} \PY{n}{np}\PY{o}{.}\PY{n}{zeros}\PY{p}{(}\PY{n}{N\PYZus{}updates}\PY{p}{)}
         \PY{n}{ar\PYZus{}vs\PYZus{}t} \PY{o}{=} \PY{n}{np}\PY{o}{.}\PY{n}{zeros}\PY{p}{(}\PY{n}{N\PYZus{}updates}\PY{p}{)}
         \PY{n}{t\PYZus{}array} \PY{o}{=} \PY{n}{np}\PY{o}{.}\PY{n}{zeros}\PY{p}{(}\PY{n}{N\PYZus{}updates}\PY{p}{)}
         \PY{n}{hr\PYZus{}vs\PYZus{}t}\PY{p}{[}\PY{l+m+mi}{0}\PY{p}{]} \PY{o}{=} \PY{n}{RHF\PYZus{}Req}
         \PY{n}{hv\PYZus{}vs\PYZus{}t}\PY{p}{[}\PY{l+m+mi}{0}\PY{p}{]} \PY{o}{=} \PY{n}{v\PYZus{}init}
         
         \PY{c+c1}{\PYZsh{}\PYZsh{}\PYZsh{} Let\PYZsq{}s get the RHF Harmonic Force Spline}
         \PY{c+c1}{\PYZsh{}\PYZsh{}\PYZsh{} use cubic spline interpolation}
         \PY{n}{order} \PY{o}{=} \PY{l+m+mi}{3}
         
         \PY{c+c1}{\PYZsh{}\PYZsh{}\PYZsh{} spline for RHF Energy}
         \PY{n}{RHF\PYZus{}Harm\PYZus{}Pot\PYZus{}Spline} \PY{o}{=} \PY{n}{InterpolatedUnivariateSpline}\PY{p}{(}\PY{n}{r\PYZus{}fine}\PY{p}{,} \PY{n}{RHF\PYZus{}Harm\PYZus{}Pot}\PY{p}{,} \PY{n}{k}\PY{o}{=}\PY{n}{order}\PY{p}{)}
         \PY{n}{RHF\PYZus{}Harm\PYZus{}Force} \PY{o}{=} \PY{n}{RHF\PYZus{}Harm\PYZus{}Pot\PYZus{}Spline}\PY{o}{.}\PY{n}{derivative}\PY{p}{(}\PY{p}{)}
         \PY{c+c1}{\PYZsh{}\PYZsh{}\PYZsh{} first Velocity Verlet update}
         \PY{n}{result\PYZus{}array} \PY{o}{=} \PY{n}{Velocity\PYZus{}Verlet}\PY{p}{(}\PY{n}{r\PYZus{}init}\PY{p}{,} \PY{n}{v\PYZus{}init}\PY{p}{,} \PY{n}{mu}\PY{p}{,} \PY{n}{RHF\PYZus{}Harm\PYZus{}Force}\PY{p}{,} \PY{n}{dt}\PY{p}{)}
         \PY{c+c1}{\PYZsh{}hresult\PYZus{}array = Velocity\PYZus{}Verlet(r\PYZus{}init, v\PYZus{}init, mu, HF, dt)}
         
         \PY{n+nb}{print}\PY{p}{(}\PY{n}{result\PYZus{}array}\PY{p}{)}
         \PY{c+c1}{\PYZsh{}print(hresult\PYZus{}array)}
         \PY{n}{ar\PYZus{}vs\PYZus{}t}\PY{p}{[}\PY{l+m+mi}{0}\PY{p}{]} \PY{o}{=} \PY{n}{harmonic\PYZus{}position}\PY{p}{(}\PY{n}{RHF\PYZus{}k}\PY{p}{,} \PY{n}{mu}\PY{p}{,} \PY{n}{v\PYZus{}init}\PY{p}{,} \PY{n}{RHF\PYZus{}Req}\PY{p}{,} \PY{l+m+mi}{0}\PY{p}{)}
         \PY{c+c1}{\PYZsh{}\PYZsh{}\PYZsh{} do the update N\PYZus{}update\PYZhy{}1 more times}
         \PY{k}{for} \PY{n}{i} \PY{o+ow}{in} \PY{n+nb}{range}\PY{p}{(}\PY{l+m+mi}{1}\PY{p}{,}\PY{n}{N\PYZus{}updates}\PY{p}{)}\PY{p}{:}
             \PY{n}{tmp} \PY{o}{=} \PY{n}{Velocity\PYZus{}Verlet}\PY{p}{(}\PY{n}{result\PYZus{}array}\PY{p}{[}\PY{l+m+mi}{0}\PY{p}{]}\PY{p}{,} \PY{n}{result\PYZus{}array}\PY{p}{[}\PY{l+m+mi}{1}\PY{p}{]}\PY{p}{,} \PY{n}{mu}\PY{p}{,} \PY{n}{RHF\PYZus{}Harm\PYZus{}Force}\PY{p}{,} \PY{n}{dt}\PY{p}{)}
             \PY{n}{result\PYZus{}array} \PY{o}{=} \PY{n}{tmp}
             \PY{n}{t\PYZus{}array}\PY{p}{[}\PY{n}{i}\PY{p}{]} \PY{o}{=} \PY{n}{dt}\PY{o}{*}\PY{n}{i}
             \PY{n}{hr\PYZus{}vs\PYZus{}t}\PY{p}{[}\PY{n}{i}\PY{p}{]} \PY{o}{=} \PY{n}{result\PYZus{}array}\PY{p}{[}\PY{l+m+mi}{0}\PY{p}{]}
             \PY{n}{hv\PYZus{}vs\PYZus{}t}\PY{p}{[}\PY{n}{i}\PY{p}{]} \PY{o}{=} \PY{n}{result\PYZus{}array}\PY{p}{[}\PY{l+m+mi}{1}\PY{p}{]}
             \PY{n}{ar\PYZus{}vs\PYZus{}t}\PY{p}{[}\PY{n}{i}\PY{p}{]} \PY{o}{=} \PY{n}{harmonic\PYZus{}position}\PY{p}{(}\PY{n}{RHF\PYZus{}k}\PY{p}{,} \PY{n}{mu}\PY{p}{,} \PY{n}{v\PYZus{}init}\PY{p}{,} \PY{n}{RHF\PYZus{}Req}\PY{p}{,} \PY{n}{dt}\PY{o}{*}\PY{n}{i}\PY{p}{)}
             \PY{c+c1}{\PYZsh{}tmp = Velocity\PYZus{}Verlet(hresult\PYZus{}array[0], hresult\PYZus{}array[1], mu, HF, dt)}
             \PY{c+c1}{\PYZsh{}hr\PYZus{}vs\PYZus{}t[i] = hresult\PYZus{}array[0]}
             \PY{c+c1}{\PYZsh{}hv\PYZus{}vs\PYZus{}t[i] = hresult\PYZus{}array[1]}
         
         \PY{c+c1}{\PYZsh{}\PYZsh{}\PYZsh{} Plot the trajectory of bondlength vs time:}
         \PY{c+c1}{\PYZsh{}plt.plot(t\PYZus{}array, r\PYZus{}vs\PYZus{}t, \PYZsq{}red\PYZsq{}, t\PYZus{}array, hr\PYZus{}vs\PYZus{}t, \PYZsq{}blue\PYZsq{})}
         \PY{n}{plt}\PY{o}{.}\PY{n}{plot}\PY{p}{(}\PY{n}{t\PYZus{}array}\PY{p}{,} \PY{n}{hr\PYZus{}vs\PYZus{}t}\PY{p}{,} \PY{l+s+s1}{\PYZsq{}}\PY{l+s+s1}{red}\PY{l+s+s1}{\PYZsq{}}\PY{p}{,} \PY{n}{t\PYZus{}array}\PY{p}{,} \PY{n}{ar\PYZus{}vs\PYZus{}t}\PY{p}{,} \PY{l+s+s1}{\PYZsq{}}\PY{l+s+s1}{b\PYZhy{}\PYZhy{}}\PY{l+s+s1}{\PYZsq{}}\PY{p}{)}
         \PY{n}{plt}\PY{o}{.}\PY{n}{show}\PY{p}{(}\PY{p}{)}
         
         \PY{c+c1}{\PYZsh{}\PYZsh{}\PYZsh{} plot the phase space trajectory of position vs momentum}
         \PY{c+c1}{\PYZsh{}plt.plot(mu*v\PYZus{}vs\PYZus{}t, r\PYZus{}vs\PYZus{}t, \PYZsq{}blue\PYZsq{}, mu*hv\PYZus{}vs\PYZus{}t, hr\PYZus{}vs\PYZus{}t, \PYZsq{}purple\PYZsq{})}
         \PY{c+c1}{\PYZsh{}plt.show()}
             
\end{Verbatim}


    \begin{Verbatim}[commandchars=\\\{\}]
[1.6975188049482695, -0.004729548408292249]

    \end{Verbatim}

    \begin{center}
    \adjustimage{max size={0.9\linewidth}{0.9\paperheight}}{output_19_1.png}
    \end{center}
    { \hspace*{\fill} \\}
    
    \begin{Verbatim}[commandchars=\\\{\}]
{\color{incolor}In [{\color{incolor}19}]:} \PY{n}{plt}\PY{o}{.}\PY{n}{plot}\PY{p}{(} \PY{n}{t\PYZus{}array}\PY{p}{,} \PY{n}{ar\PYZus{}vs\PYZus{}t}\PY{p}{,} \PY{l+s+s1}{\PYZsq{}}\PY{l+s+s1}{b\PYZhy{}\PYZhy{}}\PY{l+s+s1}{\PYZsq{}}\PY{p}{)}
         \PY{n}{plt}\PY{o}{.}\PY{n}{show}\PY{p}{(}\PY{p}{)}
\end{Verbatim}


    \begin{center}
    \adjustimage{max size={0.9\linewidth}{0.9\paperheight}}{output_20_0.png}
    \end{center}
    { \hspace*{\fill} \\}
    
    Now that we have our initial conditions chosen, our force as a function
of separation known, and our Velocity Verlet function completed, we are
ready to run our simulations!

    \begin{Verbatim}[commandchars=\\\{\}]
{\color{incolor}In [{\color{incolor}22}]:} \PY{c+c1}{\PYZsh{}\PYZsh{}\PYZsh{} how many updates do you want to perform?}
         \PY{n}{N\PYZus{}updates} \PY{o}{=} \PY{l+m+mi}{200000}
         
         \PY{c+c1}{\PYZsh{}\PYZsh{}\PYZsh{} establish time\PYZhy{}step for integration to be 0.02 atomic units... this is about 0.0005 femtoseconds}
         \PY{c+c1}{\PYZsh{}\PYZsh{}\PYZsh{} so total time is 200000*0.02 atomic units of time which is \PYZti{}9.6e\PYZhy{}13 s, or 960 fs}
         \PY{n}{dt} \PY{o}{=} \PY{l+m+mf}{0.02}
         
         \PY{c+c1}{\PYZsh{}\PYZsh{}\PYZsh{} Now use r\PYZus{}init and v\PYZus{}init and run velocity verlet update N\PYZus{}updates times, plot results}
         \PY{c+c1}{\PYZsh{}\PYZsh{}\PYZsh{} these arrays will store the time, the position vs time, and the velocity vs time}
         \PY{n}{r\PYZus{}vs\PYZus{}t} \PY{o}{=} \PY{n}{np}\PY{o}{.}\PY{n}{zeros}\PY{p}{(}\PY{n}{N\PYZus{}updates}\PY{p}{)}
         \PY{n}{v\PYZus{}vs\PYZus{}t} \PY{o}{=} \PY{n}{np}\PY{o}{.}\PY{n}{zeros}\PY{p}{(}\PY{n}{N\PYZus{}updates}\PY{p}{)}
         \PY{n}{t\PYZus{}array} \PY{o}{=} \PY{n}{np}\PY{o}{.}\PY{n}{zeros}\PY{p}{(}\PY{n}{N\PYZus{}updates}\PY{p}{)}
         
         \PY{c+c1}{\PYZsh{}\PYZsh{}\PYZsh{} harmonic results}
         \PY{n}{hr\PYZus{}vs\PYZus{}t} \PY{o}{=} \PY{n}{np}\PY{o}{.}\PY{n}{zeros}\PY{p}{(}\PY{n}{N\PYZus{}updates}\PY{p}{)}
         \PY{n}{hv\PYZus{}vs\PYZus{}t} \PY{o}{=} \PY{n}{np}\PY{o}{.}\PY{n}{zeros}\PY{p}{(}\PY{n}{N\PYZus{}updates}\PY{p}{)}
         
         
         \PY{c+c1}{\PYZsh{}\PYZsh{}\PYZsh{} first entry is the intial position and velocity}
         \PY{n}{r\PYZus{}vs\PYZus{}t}\PY{p}{[}\PY{l+m+mi}{0}\PY{p}{]} \PY{o}{=} \PY{n}{r\PYZus{}init}
         \PY{n}{v\PYZus{}vs\PYZus{}t}\PY{p}{[}\PY{l+m+mi}{0}\PY{p}{]} \PY{o}{=} \PY{n}{v\PYZus{}init}
         
         \PY{n}{hr\PYZus{}vs\PYZus{}t}\PY{p}{[}\PY{l+m+mi}{0}\PY{p}{]} \PY{o}{=} \PY{n}{r\PYZus{}init}
         \PY{n}{hv\PYZus{}vs\PYZus{}t}\PY{p}{[}\PY{l+m+mi}{0}\PY{p}{]} \PY{o}{=} \PY{n}{v\PYZus{}init}
         
         \PY{c+c1}{\PYZsh{}\PYZsh{}\PYZsh{} first Velocity Verlet update}
         \PY{n}{result\PYZus{}array} \PY{o}{=} \PY{n}{Velocity\PYZus{}Verlet}\PY{p}{(}\PY{n}{r\PYZus{}init}\PY{p}{,} \PY{n}{v\PYZus{}init}\PY{p}{,} \PY{n}{mu}\PY{p}{,} \PY{n}{RHF\PYZus{}Force}\PY{p}{,} \PY{n}{dt}\PY{p}{)}
         \PY{c+c1}{\PYZsh{}hresult\PYZus{}array = Velocity\PYZus{}Verlet(r\PYZus{}init, v\PYZus{}init, mu, HF, dt)}
         
         \PY{n+nb}{print}\PY{p}{(}\PY{n}{result\PYZus{}array}\PY{p}{)}
         \PY{c+c1}{\PYZsh{}print(hresult\PYZus{}array)}
         
         \PY{c+c1}{\PYZsh{}\PYZsh{}\PYZsh{} do the update N\PYZus{}update\PYZhy{}1 more times}
         \PY{k}{for} \PY{n}{i} \PY{o+ow}{in} \PY{n+nb}{range}\PY{p}{(}\PY{l+m+mi}{1}\PY{p}{,}\PY{n}{N\PYZus{}updates}\PY{p}{)}\PY{p}{:}
             \PY{n}{tmp} \PY{o}{=} \PY{n}{Velocity\PYZus{}Verlet}\PY{p}{(}\PY{n}{result\PYZus{}array}\PY{p}{[}\PY{l+m+mi}{0}\PY{p}{]}\PY{p}{,} \PY{n}{result\PYZus{}array}\PY{p}{[}\PY{l+m+mi}{1}\PY{p}{]}\PY{p}{,} \PY{n}{mu}\PY{p}{,} \PY{n}{RHF\PYZus{}Force}\PY{p}{,} \PY{n}{dt}\PY{p}{)}
             \PY{n}{result\PYZus{}array} \PY{o}{=} \PY{n}{tmp}
             \PY{n}{t\PYZus{}array}\PY{p}{[}\PY{n}{i}\PY{p}{]} \PY{o}{=} \PY{n}{dt}\PY{o}{*}\PY{n}{i}
             \PY{n}{r\PYZus{}vs\PYZus{}t}\PY{p}{[}\PY{n}{i}\PY{p}{]} \PY{o}{=} \PY{n}{result\PYZus{}array}\PY{p}{[}\PY{l+m+mi}{0}\PY{p}{]}
             \PY{n}{v\PYZus{}vs\PYZus{}t}\PY{p}{[}\PY{n}{i}\PY{p}{]} \PY{o}{=} \PY{n}{result\PYZus{}array}\PY{p}{[}\PY{l+m+mi}{1}\PY{p}{]}
             \PY{c+c1}{\PYZsh{}tmp = Velocity\PYZus{}Verlet(hresult\PYZus{}array[0], hresult\PYZus{}array[1], mu, HF, dt)}
             \PY{c+c1}{\PYZsh{}hr\PYZus{}vs\PYZus{}t[i] = hresult\PYZus{}array[0]}
             \PY{c+c1}{\PYZsh{}hv\PYZus{}vs\PYZus{}t[i] = hresult\PYZus{}array[1]}
         
         \PY{c+c1}{\PYZsh{}\PYZsh{}\PYZsh{} Plot the trajectory of bondlength vs time:}
         \PY{c+c1}{\PYZsh{}plt.plot(t\PYZus{}array, r\PYZus{}vs\PYZus{}t, \PYZsq{}red\PYZsq{}, t\PYZus{}array, hr\PYZus{}vs\PYZus{}t, \PYZsq{}blue\PYZsq{})}
         \PY{n}{plt}\PY{o}{.}\PY{n}{plot}\PY{p}{(}\PY{n}{t\PYZus{}array}\PY{p}{,} \PY{n}{r\PYZus{}vs\PYZus{}t}\PY{p}{)}
         \PY{n}{plt}\PY{o}{.}\PY{n}{show}\PY{p}{(}\PY{p}{)}
         
         \PY{c+c1}{\PYZsh{}\PYZsh{}\PYZsh{} plot the phase space trajectory of position vs momentum}
         \PY{c+c1}{\PYZsh{}plt.plot(mu*v\PYZus{}vs\PYZus{}t, r\PYZus{}vs\PYZus{}t, \PYZsq{}blue\PYZsq{}, mu*hv\PYZus{}vs\PYZus{}t, hr\PYZus{}vs\PYZus{}t, \PYZsq{}purple\PYZsq{})}
         \PY{c+c1}{\PYZsh{}plt.show()}
             
             
             
\end{Verbatim}


    \begin{Verbatim}[commandchars=\\\{\}]
[2.4722356719182224, -2.9065006058490773e-05]

    \end{Verbatim}

    \begin{center}
    \adjustimage{max size={0.9\linewidth}{0.9\paperheight}}{output_22_1.png}
    \end{center}
    { \hspace*{\fill} \\}
    

    % Add a bibliography block to the postdoc
    
    
    
    \end{document}
